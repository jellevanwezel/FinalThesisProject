\documentclass{article}
\usepackage{graphicx}

\begin{document}

\title{Goals KB FTP}
\author{Jelle van Wezel}

\maketitle

\section{Goal 1, Pipeline durability prediction}

The first goal of the project is to predict in some way the durability of the current gas pipes hold by Cogas.

With the data we have the following techniques can be applied.

\subsection{Regression}

The following regression techniques were selected.

\begin{itemize}
\item Linear Regression
\item Sin Regression
\item Non-linear Regression
\item Auto Regression
\end{itemize}

\subsection{Learning Vector Quantization}

With the help of a sliding window we can use LVQ to predict a future trend.

\begin{itemize}
\item LVQ with the Linear regression as label
\item LVQ with binning as label
\end{itemize}

Both LVQ methods can be done on: the whole dataset, measuring points (mps), and cathodic protection area (cpa).

\subsection{Feature Extraction}

\begin{itemize}
\item PCA
\item Suvrel
\item GML
\item Covariance Matrix
\item RGLVQ and GMLVQ, look at the matrix / vector
\end{itemize}

\section{Goal 2, Suggesting new anode location}

The second goal is the suggest a location for a new anode such that the worst parts of a existing cathodic protection (cp) area can be better protected.

\subsection{Questions}

\begin{itemize}
\item Is this even possible?
\item Are there now cp areas with multiple anodes?
\item How would this be implemented? (create 2 new cp areas?)
\end{itemize}

\subsection{Clustering}

A method of finding a location could be to cluster the mps and find the clusters with the lowest voltage readings/trends, these areas could then be suitable for a new anode.

\begin{itemize}
\item K-means
\item Neural Gas
\end{itemize}

\end{document}